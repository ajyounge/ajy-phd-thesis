%% Author: Andrew J. Younge
%% PhD Thesis/Project



%$$$$$$$$$$$$$$$$$$$$$$$$$$$$$$$$$$$$$$$$$$$$$$$$$$$$$$$$$$$$$$$$$$$$%
\chapter{Supporting High Performance Molecular Dynamics in Virtualized Clusters using IOMMU, SR-IOV, and GPUDirect}
\label{chap:mdsimulations}
%$$$$$$$$$$$$$$$$$$$$$$$$$$$$$$$$$$$$$$$$$$$$$$$$$$$$$$$$$$$$$$$$$$$$%

\begin{comment}
\begin{abstract}

Cloud infrastructure-as-a-Service paradigms have recently shown their utility for a vast array of computational problems, ranging from advanced web service architectures to high throughput computing.  However, many scientific computing applications have been slow to adapt to virtualized cloud frameworks. This is due to performance impacts of virtualization technologies, coupled with the lack of advanced hardware support necessary for running many high performance scientific a:tabnpplications at scale. 

By using KVM virtual machines that leverage both Nvidia GPUs and InfiniBand, we show that molecular dynamics simulations with LAMMPS and HOOMD run at near-native speeds. This experiment also illustrates how virtualized environments can support the latest parallel computing paradigms, including both MPI+CUDA and new GPUDirect RDMA functionality. Specific findings show initial promise in scaling of such applications to larger production deployments targeting large scale computational workloads.  


%While these experiments do go beyond a single-node, their early implementation is limited to only 4 nodes due to the lack of feasible resources. Currently efforts are under way to scale the deployment to hundreds of cores and 32 GPUs within the FutureGrid testbed, which we look to demonstrate at Supercomputing 2014 in November. 
 

%and Cloud IaaS platforms may be well suited for supporting larger scale scientific applications, including support for the long tail of science. 

%* High performance Cloud IaaS \\
%* Support mid-tier scientific computing \\
%* Long tail of science \\
%* MD simulations \\
%* Good results \\

\end{abstract}
\end{comment}

\section{Introduction}

At present we stand at the inevitable intersection between High Performance Computing (HPC) and clouds. Various platform tools such as Hadoop and MapReduce, among others, have already percolated into data intensive computing within HPC~\cite{Jha2014apache}.  In addition, there are efforts to support traditional HPC-centric scientific computing applications in virtualized cloud infrastructure.  There are a multitude of reasons for supporting parallel computation in the cloud\cite{Armbrust2010}, including features such as dynamic scalability, specialized operating environments, simple management interfaces, fault tolerance, and enhanced quality of service, to name a few. The growing importance of supporting advanced scientific computing using virtualized infrastructure can be seen by a variety of new efforts, including the NSF-funded Comet resource part of XSEDE at San Diego Supercomputer Center \cite{moore2014gateways}.  

Nevertheless, there exists a past notion that virtualization used in today's cloud infrastructure is inherently inefficient.  Historically, cloud infrastructure has also done little to provide the necessary advanced hardware capabilities that have become almost mandatory in supercomputers today, most notably advanced GPUs and high-speed, low-latency interconnects.  The result of these notions has hindered the use of virtualized environments for parallel computation, where performance must be paramount.

A growing effort is currently underway that looks to systematically identify and
reduce any overhead in virtualization technologies. This effort has, thus far,
proven to be a qualified success~\cite{Younge2011cloud, Luszczek:2011:EHC}, though further research is needed to address
issues of scalability and I/O.  Thus, we see a constantly diminishing overhead
with virtualization, not only with traditional cloud workloads
\cite{huber2011evaluating} but also with HPC workloads.  While virtualization
will almost always include some additional overhead in relation to its dynamic
features, the eventual goal for supporting HPC in virtualized environments is to
minimize what overhead exists whenever possible.  To advance the placement of
HPC applications on virtual machines, new efforts are emerging which focus
specifically on key hardware now commonplace in supercomputers. By leveraging
new virtualization tools such as IOMMU device passthrough and SR-IOV, we can now
support the such advanced hardware as the latest Nvidia Tesla GPUs
\cite{Walters2014cloud}  as well as InfiniBand fabrics for high performance networking
and I/O~\cite{jose2013sr,Musleh2014cloud}.  

With the advances in hypervisor performance coupled with the newfound availability of HPC hardware in virtual machines analogous to the most powerful supercomputers used today, we see can see the possibility of a high performance cloud infrastructure using virtualization. While our previous efforts in this area have focused on single-node advancements, it is now imperative to ensure real-world applications can also operate in distributed environments as found in today's cluster and cloud infrastructures. 

Efforts to improve power efficiency and performance in data centers has led to more heterogeneous architectures. That move toward heterogeneity has, in turn, led to support for heterogeneity in the cloud. For example, Amazon EC2 supports GPU accelerators in EC2~\cite{www-amazon-gpu}, and OpenStack supports heterogeneity using flavors~\cite{www-openstack-flavors}. These advancements in cloud-level support for heterogeneity combined with better support for high-performance virtualization makes the use of cloud for HPC much more feasible for a wider range of applications and platforms.

In this paper we discribe background a related work. Then, we describe a heterogeneous cloud platform, based on OpenStack. This
effort has been under development at USC/ISI since 2011~\cite{crago2011heterogeneous}.
We describe our work towards integrating GPU and InfiniBand support into
OpenStack, and we describe the heterogeneous scheduling additions that are
necessary to support not only attached accelerators, but any cloud composed of
heterogeneous elements.  

We then demonstrate running two molecular dynamics simulations, LAMMPS and HOOMD, in a virtual infrastructure complete with both Kepler GPUs and QDR InfiniBand.  Both HOOMD and LAMMPS are used extensively in some of the world's fastest supercomputers and represent example simulations that HPC supports today.  We show that these applications are able to run at near-native speeds within a completely virtualized environment, demonstrating just small performance impacts that are usually acceptable by many users. Furthermore, we demonstrate the ability of such a virtualized environment to support cutting edge software tools such as RDMA GPUDirect, illustrating how cutting-edge HPC technologies are also possible in a virtualized environment. 

Following these efforts, we hope to ensure upstream infrastructure projects such as OpenStack \cite{www-Openstack, pepple2011deploying} are able to make effective and quick use of these features, allowing users to build private cloud infrastructure to support high performance distributed computational workloads. 

%Furthermore, the tighet and exact integration into an open source Cloud infrastructure framework such as OpenStack also becomes a critical next step.  

%This manuscript demonstrates 


%The tight and exact integration into an open source cloud IaaS framework such as OpenStack \cite{www-openstack} becomes critical.

 

%* Broadly talk about clouds, OpenStack, some of our earlier HPC and GPU work\\
%* We've shown single node GPU performance at nearly 100\% efficieny (we'll need to be more accurate/precise than that in the actual submission). \\
%* In this work we demonstrate two molecular dynamics simuations running in a virtual infrastructure: LAMMPS and  HOOMD \\
%* We show that both perform near-native, and we show GPU Direct RDMA for the first time in the cloud \\

\section{Background and Related Work}

%NOTE: first introduce virtualization, and I/O featuresets. Then enable GPUs. Then bring in SR-IOv and infiniBand. Finally, discuss applications and GPUDirect.

Virtualization technologies and hypervisors have been seen widespread deployment in support of a vast array of applications.  This ranges from public commercial Cloud deployments such as Amazon EC2 \cite{hazelhurst2008scientific,amazon2010}, Microsoft Azure \cite{jennings2010cloud}, and Google's Cloud Platform \cite{www-google-platform} to private deployments within colocation facilities, corporate data centers, and even national scale cyber infrastructure initiatives.  All these support look to support various use cases and applications such as web servers, ACID and BASE databases, online object storage, and even distributed systems, to name a few.  

The use of virtualization and hypervisors specifically support various HPC solutions has been studied with mixed results.  In ~\cite{Younge2011cloud}, it is found that there is a great deal of variance between hypervisors when running various distributed memory and MPI applications, finding that KVM overall performed well across an array of HPC benchmarks.  Furthermore, some applications may not may fit well into default virtualized environments, such as High Performance Linpack \cite{Luszczek:2011:EHC}. Other studies have specifically looked at interconnect performance in virtualization and found the best-case scenario to be lacking \cite{Ramakrishnan2012} with up to 60\% performance penalties with conventional techniques.
 
Recently, various CPU architectures have added support for I/O virtualization mechanisms in the CPU ISA through the use of an I/O memory management unit (IOMMU). Often, this is referred to as PCI Passthrough, as it enabled devices on the PCI-Express bus to be passed directly to a specific virtual machine (VM).  Specific hardware implementations include Intel's VT-d \cite{intelvirtualization}, AMD's IOMMU \cite{amdiommu} from x86\_64 architectures, and even more recently ARM System MMU \cite{armmmu}.  All of these implementations effectively look to aid in the usage of DMA-capable hardware to be used within a specific virtual machine. Using these features, a wide array of hardware can be utilized directly within VMs and enable fast and efficient computation and I/O capabilities.

With PCI Passthrough, a PCI device is handed directly to a running (or booting) VM, thereby relinquishing control of the device within the host entirely. This is different from typical VM usage where hardware is emulated in the host and used in a guest VM, such as with bridged ethernet adapters or emulated VGA devices. Performing PCI Passthrough requires the host to seize the device upon boot using a specialized driver to effectively block normal driver initialization. In the instance of the KVM hypervisor, this is done using the \emph{vfio} and \emph{pci\_stub} drivers. Then, this driver relinquishes control to the VM, whereby normal device drivers initiate the hardware and enable the device for use by the guest OS.  

\subsection{GPU Passthrough}

Nvidia GPUs comprise the single most common accelerator in the Nov 2014 Top 500 List \cite{www-top500} and represent an increasing shift towards accelerators for HPC applications. Historically, GPU usage in a virtualized environment has been difficult, especially for scientific computation. Various front-end remote API implementations have been developed to provide CUDA and OpenCL libraries in VMs, which translate library calls to a back-end or remote GPU. One common use case of this is rCUDA \cite{duato2011enabling}, which provides a front-end CUDA API within a VM or any compute node, and then sends the calls via Ethernet or InfiniBand to a separate node with 1 or more GPUs. While this method is valid, it has the drawback of relying on the interconnect itself and the bandwidth available, which can be especially problematic on Ethernet. Furthermore, as this method consumes bandwidth, it can leave little remaining for MPI or RDMA routines, thereby constructing a bottleneck for some MPI+CUDA applications that depend on inter-process communication.

Recently efforts have been seen to support such GPU accelerators within VMs using IOMMU technologies, with implementations now available with KVM \cite{Walters2014cloud}, Xen \cite{Younge2014hpgc} and VMWare \cite{Vu2014}.  These efforts have shown that GPUs can achieve up to 99\% of their bare metal performance when passed to a virtual machine using PCI Passthrough.  VMWare specifically shows how the such PCI Passthrough solutions perform well and are likely to outperform front-end Remote API solutions such as rCUDA within a VM\cite{Vu2014}.  While these works demonstrate PCI Passthrough performance across a range of hypervisors and GPUs, they have been limited to investigating single node performance until now. 

\subsection{SR-IOV and InfiniBand}

With almost all parallel HPC applications, the interconnect fabric which enables fast and efficient communication between processors becomes a central requirement to achieving good performance. Specifically, a high bandwidth link is needed for distributed processors to share large amounts of data across the system. Furthermore, low latency becomes equally important for ensuring quick delivery of small message communications and resolving large collective barriers within many parallelized codes. One such interconnect, InfiniBand, has become the most common implementation used within the Top500 list. However previously InfiniBand was inaccessible to virtualized environments.  

Supporting I/O interconnects in VMs has been aided by Single Root I/O Virtualization (SR-IOV), whereby multiple virtual PCI functions are created in hardware to represent a single PCI device. These virtual functions (VFs) can then be passed to a VM and used as by the guest as if it had direct access to that PCI device. SR-IOV allows for the virtualization and multiplexing to be done within the hardware, effectively providing higher performance and greater control than software solutions. 

SR-IOV has been used in conjunction with Ethernet devices to provide high performance 10Gb TCP/IP connectivity within VMs \cite{Liu2010}, offering near-native bandwidth and advanced QoS features not easily obtained through emulated Ethernet offerings. Currently Amazon EC2 offers a high performance VM solution utilizing SR-IOV enabled 10Gb Ethernet adapters. While SR-IOV enabled 10Gb Ethernet solutions offers a big forward in performance, Ethernet still does not offer the high bandwidth or low latency typically found with InfiniBand solutions. 

Recently SR-IOV support for InfiniBand has been added by Mellanox in the ConnectX series adapters. Initial evaluation of SR-IOV InfiniBand within KVM VMs has proven has found point-to-point bandwidth to be near-native, but up to 30\% latency overhead for very small messages \cite{jose2013sr, RuivoAGTKNR14}. However, even with the noted overhead, this still signifies up to an order of magnitude difference in latency between InfiniBand and Ethernet with VMs. Furthermore, advanced configuration of SR-IOV enabled InfiniBand fabric is taking shape, with recent research showing up to a 30\% reduction in the latency overhead \cite{Musleh2014cloud}. However, real application performance has not yet been well understood until now. 

\subsection{GPUDirect}
NVIDIA's GPUDirect technology was introduced to reduce the overhead of data
movement across GPUs~\cite{GPUDirect, shainer2011development}.  GPUDirect
supports both networking as
well as peer-to-peer interfaces for single node multi-GPU systems.  The most
recent implementation of GPUDirect, version 3, adds support for RDMA over
InfiniBand for Kepler-class GPUs.

The networking component of GPUDirect relies on three key technologies: CUDA 5
(and up), a CUDA-enabled MPI implementation, and a Kepler-class GPU (RDMA only).
Both MVAPICH and OpenMPI support GPUDirect.  Support for RDMA over GPUDirect is
enabled by the MPI library, given supported hardware, and does not depend on
application-level changes to a user's code.

In this paper, our GPUDirect work focuses on GPUDirect v3 for multi-node RDMA
support.  We demonstrate scaling for up to 4 nodes connected via QDR InfiniBand
and show that GPUDirect RDMA improves both scalability and overall performance
by approximately 9\% at no cost to the end user.

\section{A Cloud for High Performance Computing}
With support for GPU Passthrough, SR-IOV, and GPUDirect, we have the building
blocks for a high performance, heterogeneous cloud.  In addition, other common
accelerators (e.g. Xeon Phi~\cite{Phi}) have similarly been demonstrated in
virtualized environments.  Our vision is of a heterogeneous cloud, supporting
both high speed networking and accelerators for tightly coupled applications.

To this end we have developed a heterogeneous cloud based on
OpenStack~\cite{www-Openstack}.  In our previous work, we
have demonstrated the ability to rapidly provision GPU, bare metal, and other
heterogeneous resources within a single cloud~\cite{crago2011heterogeneous}.
Building on this effort we have added support for GPU passthrough to OpenStack
as well as SR-IOV support for both ConnectX-2 and ConnectX-3 Infiniband devices.
Mellanox separately supports an OpenStack InfiniBand networking plugin for
OpenStack's Neutron service~\cite{ML2}, however the Mellanox plugin depends on
the ConnectX-3 adapter.  Our institutional requirements depend on ConnecteX-2
SR-IOV support, requiring an independent implementation.

OpenStack supports services for networking (Neutron), compute (Nova), identity
(Keystone), storage (Cinder, Swift), and others.  Our work focuses entirely
on the compute service.  

Scheduling is implemented at two levels: the cloud-level and the node-level.  In
our earlier work, we have developed a cloud-level heterogeneous scheduler for OpenStack,
allowing scheduling based on architectures and
resources~\cite{crago2011heterogeneous}.  In this model, the cloud-level
scheduler dispatches jobs to nodes based on resource requirements (e.g. Kepler
GPU) and node-level resource availability.

At the node, a second level of scheduling occurs to ensure that resources are
tracked and not over-committed.  Unlike traditional cloud paradigms, devices
passed into VMs cannot be over-committed.  We treat devices, whether GPUs or
InfiniBand virtual functions, as schedulable resources.  Thus, it is the responsibility of the
individual node to track resources committed and report availability to the
cloud-level scheduler.  For reporting, we piggyback on top of OpenStack's
existing reporting mechanism to provide a low overhead solution.


\section{Benchmarks}
We selected two molecular dynamics (MD) applications for evaluation in this study:
LAMMPS and HOOMD~\cite{plimpton2007lammps,anderson2010hoomd}.  These MD simulations are chosen to represent a subset of advance parallel computation for a number of fundamental reasons:

\begin{itemize}
\item MD simulations provide a practical representation of N-Body simulations, which is one of the major computational \emph{Dwarfs} \cite{asanovic2006landscape} in parallel and distributed computing. 
\item MD simulations are one of the most widely deployed applications on large scale supercomputers today.
\item Many MD simulations have a hybrid MPI+CUDA programming model, which has often become commonplace in HPC as the use of accelerators increases.
\end{itemize}

As such, we look to LAMMPS and HOOMD to provide a real-world example for running cutting-edge parallel programs on virtualized infrastructure. While these applications by no means represent all parallel scientific computing efforts (as justified by the 13 Dwarfs defined in \cite{asanovic2006landscape}), we hope these MD simulators offer a more pragmatic viewpoint than traditional synthetic HPC benchmarks such as High Performance Linpack. 

\paragraph {LAMMPS} The Large-scale Atomic/Molecular Parallel Simulator is a
well-understood highly parallel molecular dynamics simulator.  It supports both
CPU and GPU-based workloads.  Unlike many simulators, both MD and otherwise,
LAMMPS is heterogeneous.  It will use both GPUs and multicore CPUs concurrently.
For this study, this heterogeneous functionality introduces additional load on
the host, allowing LAMMPS to utilize all available cores on a given system.
Networking in LAMMPS is accomplished using a typical MPI model. That is, data is
copied from the GPU back to the host and sent over the InfiniBand fabric.  No
RDMA is used for these experiments.  

\paragraph{HOOMD-blue} The Highly Optimized Object-oriented Many-particle
Dynamics -- Blue Edition is a particle dynamics simulator capable of
scaling into the thousands of GPUs.  HOOMD supports executing on both CPUs and
GPUs.  Unlike LAMMPS, HOOMD is homogeneous and does not support mixing
of GPUs and CPUs.  HOOMD supports GPUDirect using a CUDA-enabled MPI.
In this paper we focus on HOOMD's
support for GPUDirect and show its benefits for increasing cluster sizes.  




%Similarly, Infiniband SR-IOV (Single Root I/O Virtualization) has been evaluated within the context of microbenchmarks~\cite{SRIOVInfiniband,Musleh2014cloud}, but performance for real applications is not yet well-understood.

%moved to related work. right placement?
%Recent work recent work has focused on single-node performance.  In \cite{walters2014}, we've shown how the latest Kepler GPUs from Nvidia  with Sandy-Bridge Intel Xeon CPUs can perform at near-native performance running various workloads across wide range of hypervisors. Furthermore, advanced configuration of SR-IOV enabled Infiniband fabric has taken shape, with recent research showing up to a 30\% reduction latency \cite{musleh2014}.  



\section{Experimental Setup}

Using two molecular dynamics tools, LAMMPS\cite{plimpton2007lammps} and HOOMD~\cite{anderson2010hoomd}, we demonstrate a high performance \textit{system}.  That is, we combine PCI passthrough for Nvidia Kepler-class GPUs with QDR Infiniband SR-IOV and show that high performance molecular dynamics simulations are achievable within a virtualized environment. 

For the first time, we also demonstrate Nvidia GPUDirect technology within such a virtual environment.  Thus, we look to not only illustrate that virtual machines provide a flexible high performance infrastructure for scaling scientific workloads including MD simulations, but also that the latest HPC features and programming environments are also available in this same model.   

\subsection{Node configuration}

To support the use of Nvidia GPUs and InfiniBand within a VM, specific and exact host configuration is needed. This node configuration is illustrated in Figure \ref{F:passthrough}.  While our implementation is specific to the KVM hypervisor, this setup represents a design that can be hypervisor agnostic.

\FIGURE{!htb}
  {images/host-pci-passthrough.png}
  {1.0}
  {Node PCI Passthrough of GPUs and InfiniBand}
  {F:passthrough}


Each node in the testbed uses CentOS 6.4 with a 3.13 upstream Linux kernel for the host OS, along with the latest KVM hypervisor, QEMU 2.1, and the \emph{vfio} driver.  Each Guest VM runs CentOS 6.4 with a stock 2.6.32-358.23.2 kernel. A Kepler GPU is passed through using PCI Passthrough and directly initiated within the VM via the Nvidia 331.20 driver and CUDA release 5.5. While this specific implementation used only a single GPU, it is also possible to include as many GPUs as one can fit within the PCI Express bus if desired. As the GPU is used by the VM, an on-board VGA device was used by the host and a standard Cirrus VGA was emulated in the guest OS. 

With using SR-IOV, the OFED drivers version 2.1-1.0.0 are used with Mellanox ConnectX-3 VPI adapter with firmware 2.31.5050.  The host driver initiates 4 VFs, one of which is passed through to the VM where the default OFED mlnx\_ib drivers are loaded.  

%The native bare-metal base system and all guest VMs are composed of a CentOS 6.4 installation with a 2.6.32-358.23.2 stock kernel, MVAPICH 2.0 GDR, and CUDA version 5.5. Each guest was allocated 20 GB of RAM and a full socket (8 cores) as well as a single InfiniBand virtual function  and 1 Kepler GPU per VM.  


%CentOS 6.4 with a 3.13 upstream Linux kernel was used as the host OS with the KVM hypervisor.  The native bare-metal base system and all guest VMs are composed of a CentOS 6.4 installation with a 2.6.32-358.23.2 stock kernel, MVAPICH 2.0 GDR, and CUDA version 5.5. Each guest was allocated 20 GB of RAM and a full socket (8 cores) as well as a single InfiniBand virtual function  and 1 Kepler GPU per VM.  

\subsection{Cluster Configuration}

Our test environment is composed of 4 servers each with a single Nvidia
Kepler-class GPU.  Two servers are equipped with K20 GPUs, while the other two
servers are equipped with K40 GPUs, demonstrating the potential for a more
heterogeneous deployment.  Each server is composed of 2 Intel Xeon E5-2670 CPUs,
48GB of DDR3 memory, and Mellanox ConnectX-3 QDR InfiniBand.  CPU sockets and
memory are split evenly between the two NUMA nodes on each system. All
InfiniBand adapters use a single Voltaire 4036 QDR switch with a software subnet
manager for IPoIB functionality.   


For these experiments, both the GPUs and InfiniBand adapters are attached to NUMA node 1 and both the guest VMs and the base system utilized identical software stacks.  Each guest was allocated 20 GB of RAM and a full socket of 8 cores, and pinned to NUMA node 1 to ensure optimal hardware usage. While all VMs are capable of login via the InfiniBand IPoIB setup, a 1Gb Ethernet network was used for all management and login tasks.  

%Our test environment is composed of 4 servers each with a single Nvidia Kepler-class GPU.  Two servers are equipped with K20 GPUs, while the other two servers are equipped with K40 GPUs, demonstrating the potenti

For a fair and effective comparison, we also use a native environment without any virtualization. This native environment employs the same hardware configuration, and like the Guest OS runs CentOS 6.4 with the stock 2.6.32-358.23.2 kernel. 

%In order to effectively test MD simulations in LAMMPS and HOOMD beyond single-node tests, ronment.  Bespin includes 4 blades, each with 2 Intel Xeron E5-2670 CPUs, 48Gb DDR3 memory, Mellanox ConnectX3 QDR InfiniBand cards, and a mixture of Nvidia Kepler series K20 and K40 GPUs.  While the effective experimental hardware allocation remains relatively low compared to production runs of either application, it does allow for a useful evaluation at a larger scale than preivously evaluated as well as a valued extrapolation to larger resources.

 
\section{Results}

In this section, we discuss the performance of both the LAMMPS and HOOMD molecular dynamics simulation tools when running within a virtualized environment. Specifically, we scale each application to 32 cores and 4 GPUs, both in a native bare-metal and virtualized environments.  Each application set was run 10 times, with the results averaged accordingly. 

\subsection{LAMMPS}

\FIGURE{!htb}
  {images/lammps-lj-scale.png}
  {1.0}
  {LAMMPS LJ Performance}
  {F:lammps-lj}


Figure~\ref{F:lammps-lj} shows one of the most common LAMMPS algorithms used; the Lennard-Jones potential (LJ).  This algorithm is deployed in two main configurations - a 1:1 core to GPU mapping, and a 8:1 core to GPU mapping.  With the LAMMPS GPU implementation, a delicate balance between GPUs and CPUs is required to find the optimal ratio for fastest computation, however here we just look at the two most obvious choices. With small problem sizes, the 1:1 mapping outperforms the more complex core deployment, as the problem does not require the additional complexity provided with multi-core solution.  As expected the multi-core configuration quickly offers better performance for larger problem sizes, achieving roughly twice the performance with all 8 available cores. This is largely due to the availability of all 8 cores to keep the GPU running 100\% with continual computation.
 
The important factor for this manuscript is the relative performance of the virtualized environment. From the results, it is clear the VM solution performs very well compared to the best-case native deployment. For the multi-core configuration across all problem sizes, the virtualized deployment averaged 98.5\% efficiency compared to native. The single core per GPU deployment reported better-than native performance at 100\% native.  This is likely due to caching effects, but further investigation is needed to fully identify this occurrence. 

 %and the Rhodopsin protein in solvated lipid bilayer benchmark (Rhodo), both running with the GPU package across 8 cores per GPU. Here we see that both benchmarks scale remarkably well in the virtualized KVM guest environment. 
%Compared to the base system performance, the VMs running LAMMPs acheive 96.7\% and 99.3\% efficiency for LJ and Rhodo, respectively when running across all nodes.  These low overheads illustrate the utility of running LAMPS on cloud infrastructure, and also hold promise for other hybrid MPI + CUDA applications to also scale well in a virtualized environment. 


Another common LAMMPS algorithm, the Rhodopsin protein in solvated lipid bilayer benchmark (Rhodo), was also run with results given in Figure \ref{F:lammps-rhodo}. As with the LJ runs, we see the multi-core to GPU configuration resulting in higher computational performance for the larger problem sizes compared to the single core per GPU configuration, as expected.  

\FIGURE{!htb}
  {images/lammps-rhodo-scale.png}
  {1.0}
  {LAMMPS RHODO Performance}
  {F:lammps-rhodo}

Again, the overhead of the virtualized configuration remains low across all configurations and problem sizes, with an average 96.4\% efficiency compared to native. Interestingly enough, we also see the performance gap decrease as the problem size increases, with the 512k problem size in yielding 99.3\% of native performance.  This finding leads us to extrapolate that a virtualized MPI+CUDA implementation would scale to a larger computational resource with similar success. 


\subsection{HOOMD}




% Compared to the base system's performance, we see overheads of 3.2\% and 0.6\% for the LJ and Rhodo benchmarks, respectively, when running 8 cores per GPU at experimental scale. 


In Figure~\ref{F:HOOMD} we show the performance of a Lennard-Jones liquid simulation with 256K particles running under HOOMD.  HOOMD includes support for CUDA-aware MPI implementations via GPUDirect.  The MVAPICH 2.0 GDR implementation enables a further optimization by supporting RDMA for GPUDirect. From Figure~\ref{F:HOOMD} we can see that HOOMD simulations, both with and without GPUDirect, perform very near-native.  The GPUDirect results at 4 nodes achieve 98.5\% of the base system's performance.  The non-GPUDirect results achieve 98.4\% efficiency at 4 nodes. These results indicate the virtualized HPC environment is able to support such complex workloads. While the effective testbed size is relatively small, it indicates that such workloads may scale equally well to hundreds or thousands of nodes. 

\FIGURE{!htb}
  {images/hoomd.png}
  {1.0}
  {HOOMD LJ Performance with 256k Simulation}
  {F:HOOMD}


\section{Discussion}

From the results, we see the potential for running HPC applications in a virtualized environment using GPUs and InfiniBand interconnect fabric. Across all LAMMPS runs with ranging core configurations, we found only a 1.9\% overhead between the KVM virtualized environment and native. For HOOMD, we found a similar 1.5\% overhead, both with and without GPU Direct. These results go against conventional wisdom that HPC workloads do not work in VMs. In fact ,we show two N-Body type simulations programmed in an MPI+CUDA implementation perform at roughly near-native performance in tuned KVM virtual machines.  

With HOOMD, we see how GPUDirect RDMA shows a clear advantage over the
non-GPUDirect implementation, achieving a 9\% performance boost in both the
native a virtualized experiments.  While GPUDirect's performance impact has been well evaluated previously \cite{GPUDirect}, it is the author's belief that this manuscript represents the first time GPUDirect has has been utilized in a virtualized environment.  

Another interesting finding of running LAMMPS and HOOMD in a virtualized environment is as workload scales from a single node to 32 cores, the overhead does not increase. These results lend credence to the notion that this solution would also work for a much larger deployment. Specifically, it would be possible to expand such computational problems to a larger deployment in FutureGrid \cite{fox2013futuregrid}, Chameleon Cloud \cite{www-chameleon}, or even the planned NSF Comet machine at SDSC, scheduled to provide up to 2 Petaflops of computational power. Effectively, these results help support the theory that a majority of HPC computations can be supported in virtualized environment with minimal overhead. 


\section{Conclusion}

With the advent of cloud infrastructure, the ability to run large-scale parallel scientific applications has become possible but limited due to both performance and hardware availability concerns. In this work we show that advanced HPC-oriented hardware such as the latest Nvidia GPUs and InfiniBand fabric are now available within a virtualized infrastructure. Our results find MPI + CUDA applications such as molecular dynamics simulations run at near-native performance compared to traditional non-virtualized HPC infrastructure, with just an averaged 1.9\% and 1.5\% overhead for LAMMPS and HOOMD, respectively. Moving forward, we show the utility of GPUDirect RDMA for the first time in a cloud environment with HOOMD.  Effectively, we look to pave the way for large-scale virtualized cloud Infrastructure to support a wide array of advanced scientific computation commonly found running on many supercomputers today.  Our efforts leverage these technologies and provide them in an open source Infrastructure-as-a-Service framework using OpenStack.  




 
