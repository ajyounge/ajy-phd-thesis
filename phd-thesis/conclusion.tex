%% Author: Andrew J. Younge
%% PhD Thesis/Project

%$$$$$$$$$$$$$$$$$$$$$$$$$$$$$$$$$$$$$$$$$$$$$$$$$$$$$$$$$$$$$$$$$$$$%
\chapter{Conclusion}
\label{chap:conc}
%$$$$$$$$$$$$$$$$$$$$$$$$$$$$$$$$$$$$$$$$$$$$$$$$$$$$$$$$$$$$$$$$$$$$%

\TODO{Give a quick 2 page recap on the work discussed. find a way to rephrase much of the introduction.}

With the advent of virtualization and the availability of virtual machines through cloud infrastructure, a paradigm shift in distributed systems has occured. Many services and applications once deployed on workstations, private servers, personal computers, and even some supercomputers have migrated to a cloud infrastructure. The reasons for this change are vast, however these reasons are not enough to support all computational challanges within such a virtualized infrastructure.

The use of tightly coupled, distributed memory applications common in High Performance Computing communities, has seen a number of problems and complications when deployed in virtualized infrasturcture.  While the reasons for this can be vast, many challanges  stem from the performance impact and overhead associated with virtualization, along with a lack of hardware necessary to support such concurrent environments. This dissertation looks to evaluate virtualization for supporting mid-tier scientific HPC applications and provide potential solutions to these issues.  

Specifically, this   


\TODO{Answer answer the research question, did you accomplish this?}


%%%%%%%%%%%%%%%%%%%%%%%%%%%%%%%%%%%%%%%%%%%%%%%%%%%%%%%%%%%%%%%%%%%%%%
\section{Impact}
\label{sec:impact}
%%%%%%%%%%%%%%%%%%%%%%%%%%%%%%%%%%%%%%%%%%%%%%%%%%%%%%%%%%%%%%%%%%%%%%

TBD


